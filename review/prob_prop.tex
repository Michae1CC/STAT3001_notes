\subsection*{Common Probabilistic Properties and Identities}

Common probabilistic properties seen from prior courses.

\subsubsection*{Probabilistic Properties}
For any random variables, the following hold.
\begin{align}
    \Em (X)                               & = \int_{0}^{\infty} \left( 1 - F(X) \right) \, dx                                                                                                              \\
    \Em \left( aX + b \right)             & = a \Em X + b                                                                                                                                                  \\
    \Em \left( g(X) + h(X) \right)        & = \Em g(X) + \Em h (X)                                                                                                                                         \\
    \Var (X)                              & = \Em X^2 - \left( \Em X \right)^2                                                                                                                             \\
    \Var (aX + b)                         & = a^2 \Var (X)                                                                                                                                                 \\
    \Cov (X,Y)                            & = \Em XY - \Em X \Em Y                                                                                                                                         \\
    \Var (X + Y)                          & = \Var (X) + \Var (Y) + 2 \Cov (X,Y)                                                                                                                           \\
    \Em [X]                               & = \Em [ \Em [X \mid Y] ]                                                                                                                                       \\
    \Var(Y)                               & = \Em [\Var(Y | X)]+\Var(\Em[Y | X])\vspace{-4mm}                                                                                                              \\
    \left| \Em (X Y) \right|^2            & \leq \Em (X^2) \Em (Y^2)                                                                                                                                       \\
    \left| \Cov (X Y) \right|^2           & \leq \Var (X) \Var (Y)                                                                                                                                         \\
    \Pm \left( A \mid B \right)           & = \frac{\Pm \left( A \cap B \right)}{\Pm \left( B \right)}                                                                                                     \\
    \tag{Bayes' Theorem} \Pm (A \mid B)   & = \frac{\Pm (B \mid A) \Pm (A)}{\Pm (B)}                                                                                                                       \\
    \Pm \left( A_1 , \ldots , A_n \right) & = \Pm \left( A_1 \right) \Pm \left( A_2 \mid A_1 \right) \Pm \left( A_3 \mid A_1 , A_2 \right) \cdots \Pm \left( A_n \mid A_1 , A_2 , \ldots , A_{n-1} \right) \\
\end{align}

Let $\Omega = \bigcup\limits_{i=1}^{n} B_{i}$ (that is $B_{i}$ partitions the sample space) then
\begin{align}
    \tag{TLoP} \Pm (A) & = \sum_{i=1}^{n} \Pm (A\mid B_i) \Pm (B_i) \label{eqn:TLoP} \\
    \tag{TLoE} \Em (A) & = \sum_{i=1}^{n} \Em (A\mid B_i) \Pm (B_i)
\end{align}
which, when \ref{eqn:TLoP} used in conjunction with Bayes' Rule gives
\begin{equation}
    \Pm (B_i \mid A) = \frac{\Pm (A\mid B_i) \Pm (B_i)}{\sum_{j=1}^{n} \Pm (A\mid B_j) \Pm (B_j)}.
\end{equation}

\vspace{1cm}

If $X_1 , X_2 , \ldots , X_n \overset{\text{iid}}{\sim} \Wn (\mu,\sigma^2)$ and $S_n = \sum_{i=1}^{n} X_i$, then for all $\varepsilon > 0$
\begin{equation}
    \tag{Weak Law of Large Numbers} \Pm \left( \left| \frac{S_n}{n} -\mu \right| \geq \epsilon \right) = 0 .
\end{equation}

\vspace{1cm}

If $X_1 , X_2 , \ldots , X_n \overset{\text{iid}}{\sim} \Wn (\mu,\sigma^2)$ and $S_n = \sum_{i=1}^{n} X_i$, then for all $x \in \RR$
\begin{equation}
    \tag{CLT} \Pm \left( \frac{S_n - n \mu}{\sigma \sqrt{n}} \right) \leq x = \Phi (x) .
\end{equation}

\vspace{1cm}

If $X$ is a random variable and $h$ is a convex function then
\begin{equation}
    \tag{Jensens Inequality} h(\Em (X)) \leq \Em (h(X)).
\end{equation}

\subsubsection*{Probabilistic Identities}

If $X_1 , \ldots , X_n \iid \Ber(p)$ then
\begin{equation}
    \sum_{i=1}^{n} X_i \sim \Bin (n,p).
\end{equation}

\vspace{1cm}

If $X \sim \Bin (n,p)$ and $Y \sim \Bin (m,p)$, then $X + Y \sim \Bin (n+m,p)$.

\vspace{1cm}

If $X \sim \Nor (\mu_X,\sigma^2_X)$ and $Y \sim \Nor (\mu_Y,\sigma^2_Y)$, then $X + Y \sim \Nor (\mu_X + \mu_Y,\sigma^2_X + \sigma^2_Y)$.

\vspace{1cm}

If $X_1 , X_2 , \ldots X_n \iid \calN \left( \mu , \sigma \right)$ then
\begin{equation}
    \sum_{i=1}^{n} X_i^2 = \chi_n^2 .
\end{equation}