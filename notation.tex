\section*{Symbols and Notation}

Matrices are capitalized bold face letters while vectors are lowercase bold face letters.
\begin{longtable}{lp{.70\textwidth}}                                                                                                                                                                                                                                                                                                                                                                                                                                                                                                                                                                                                                                                                                                                                                                                                                                                                                                                                                                                   \\\bottomrule
    \hline
    \emph{Syntax}                                     & \emph{Meaning}                                                                                                                                                                                                                                                                                                                                                                                                                                                                                                                                                                                                                                                                                                                                                                                                                                                                                                                                    \\\midrule
    $\triangleq$                                      & An equality which acts as a statement                                                                                                                                                                                                                                                                                                                                                                                                                                                                                                                                                                                                                                                                                                                                                                                                                                                                                                             \\\\
    $\left| \bm{A} \right|$                           & The determinate of a matrix.                                                                                                                                                                                                                                                                                                                                                                                                                                                                                                                                                                                                                                                                                                                                                                                                                                                                                                                      \\\\
    $\bm{x}^{\intercal}, \bm{X}^{\intercal}$          & The transpose operator.
    \\\\
    $\bm{x}^{\ast}, \bm{X}^{\ast}$                    & The hermitian operator.
    \\\\
    $\bm{a} .\ast \bm{b}$ or $\bm{A} .\ast \bm{B}$    & Element-wise vector (matrix) multiplication, similar to Matlab.
    \\\\
    $\propto$                                         & Proportional to.
    \\\\
    $\nabla$ or $\nabla_{\bm{f}}$                     & The partial derivative (with respect to $\bm{f}$).
    \\\\
    $\nabla \nabla$ or $H(f)$                         & The Hessian.
    \\\\
    $\sim$                                            & Distributed according to, example $X \sim \calN \left( 0,1 \right)$
    \\\\
    $\overset{\text{iid}}{\sim}$                      & Identically and independently distributed according to, example $X_1 , X_2 , \ldots X_n \sim \calN \left( 0,1 \right)$
    \\\\
    $\bm{0}$ or $\bm{0}_{n}$ or $\bm{0}_{n \times m}$ & The zero vector (matrix) of appropriate length (size) or the zero vector of length $n$ or the zero matrix with dimensions $n \times m$.
    \\\\
    $\bm{1}$ or $\bm{1}_{n}$ or $\bm{1}_{n \times m}$ & The one vector (matrix) of appropriate length (size) or the one vector of length $n$ or the one matrix with dimensions $n \times m$.
    \\\\
    $\Id_{n \times m}$                                & The matrix with ones along the diagonal and zeros on off diagonal elements.
    \\\\
    $\bm{A}_{(\cdot,\cdot)}$                          & Index slicing to extract a submatrix from the elements of $\bm{A} \in \RR^{n \times m}$, similar to indexing slicing from the python and Matlab programming languages. Each parameter can receive a single value or a 'slice' consisting of a start and an end value separated by a semicolon. The first and second parameter describe what row and columns should be selected, respectively. A single value means that only values from the single specified row/column should be selected. A slice tells us that all rows/columns between the provided range should be selected. Additionally if now start and end values are specified in the slice then all rows/columns should be selected. For example, the slice $\bm{A}_{(1:3,j:j')}$ is the submatrix $\RR^{3 \times (j' - j + 1)}$ matrix containing the first three rows of $\bm{A}$ and columns $j$ to $j'$. As another example, $\bm{A}_{(:,j)}$ is the $j^{th}$ column of $\bm{A}$.
    \\\\
    $\bm{A}^{\dagger}$                                & Denotes the unique psuedo inverse or Moore-Penore inverse of $\bm{A}$.
    \\\\
    $\CC$                                             & The complex numbers.
    \\\\
    $\operatorname{diag} \left( \bm{w} \right)$       & Vector argument, a diagonal matrix containing the elements of vector $\bm{w}$.
    \\\\
    $\operatorname{diag} \left( \bm{W} \right)$       & Matrix argument, a vector containing the diagonal elements of the matrix $\bm{W}$.
    \\\\
    $\EE$ or $\EE_{q(x)} \left[ z(x) \right]$         & Expectation, or expectation of $z(x)$ where $x \sim q(x)$.
    \\\\
    $\RR$                                             & The real numbers.
    \\\\
    $\operatorname{tr} \left( \bm{A} \right)$         & The trace of a matrix.
    \\\\
    $\VV$ or $\VV_{q(x)} \left[ z(x) \right]$         & Variance, the variance of $z(x)$ when $x \sim q(x)$.
    \\\\
    $\ZZ$                                             & The integers, $\ZZ = \left\{ \ldots ,-2,-1,0,1,2,\ldots \right\}$.
    \\\\
    $\Omega$                                          & The sample space.

    \\\bottomrule
    \hline
\end{longtable}
