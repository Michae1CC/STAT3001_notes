\subsection*{Methods of Finding Estimates Introduction}

\begin{defe}[Statistic] \label{defe: statistic}
    Let $X_1, \ldots , X_n$ be a random sample of size $n$ from a population and let $T(x_1, \ldots , x_n)$ be a real-valued or vector-valued function whose domain includes the sample space of $(X_1, \ldots , X_n)$. The the random variable or random vector $Y = T(X_1, \ldots , X_n)$ is called a {\bf statistic}. The probability distribution of a statistic $Y$ is called the {\bf sampling distribution} of $Y$ \cite{CasellaGeorge2001SI}*{page 211}.
\end{defe}

\begin{defe}[Sample Mean] \label{defe: sample_mean}
    The {\bf sample mean} is the arthicmetic average of the values in a random sample. It is usually denoted by
    \begin{equation}
        \overline{X} = \frac{1}{n} \sum_{i=1}^{n} X_i
    \end{equation}
    \cite{CasellaGeorge2001SI}*{page 212}.
\end{defe}

\begin{defe}[Sample Variance and Standard Deviation] \label{defe: var_std_dev}
    The {\bf sample variance} is the statistic defined by
    \begin{equation}
        S^2 = \frac{1}{n-1} \sum_{i=1}^{n} (X_i - \overline{X})^2.
    \end{equation}
    The {\bf sample standard deviation} is the statistic defined by $S = \sqrt{S^2}$ \cite{CasellaGeorge2001SI}*{page 212}.
\end{defe}

\begin{defe}[Sufficient Statistic] \label{defe: sufficient_statistic}
    A statistic $T(\bm{X})$ is a {\bf sufficient statistic} for $\theta$ if the conditional distribution of the sample $\bm{X}$ given the value of $T(\bm{X})$ does not depend on $\theta$ \cite{CasellaGeorge2001SI}*{page 272}.
\end{defe}

\begin{thm} \label{thm: sufficient_statistic_ratio}
    If $p(\bm{x} \mid \theta)$ is the joint pdf or pmf of $\bm{X}$ and $q(\theta \mid \theta)$ is the pdf or pmf of $T(\bm{X})$, then $T(\bm{X})$ is a sufficient statistic for $\theta$ if, for every $\bm{x}$ in the sample space, the ratio $p(\bm{x} \mid \theta) / q(T(\bm{x}) \mid \theta)$ is a constant function of $\theta$ \cite{CasellaGeorge2001SI}*{page 274}.
\end{thm}

\begin{thm}[Factorization Theorem] \label{thm: factorization_theorem}
    Let $f(\bm{x} \mid \theta)$ denote the joint pdf or pmf of a sample $\bm{X}$. A statistic $T(\bm{X})$ is a sufficient statistic for $\theta$, if and only if there exist function $g(t\mid \theta)$ and $h(\bm{x})$ such that, for all sample points $\bm{x}$ and all parameter points $\theta$,
    \begin{equation*}
        f(\bm{x} \mid \theta) = g(T(\bm{x}) \mid \theta) h(\bm{x})
    \end{equation*}
    \cite{CasellaGeorge2001SI}*{page 276}.
\end{thm}

\begin{exam}[Uniform Sufficient Statistic] \label{exam: uni_ss_p1}
    Example taken from \cite{CasellaGeorge2001SI}*{page 277} and can also be found on tutorial sheet 3. Let $X_1 , \ldots , X_n$ be iid observations from the discrete uniform distribution on $1 , \ldots , \theta$. That is, the unknown parameter, $\theta$, is a positive integer and the pmf of $X_i$ is
    \begin{equation*}
        f(x \mid \theta) =
        \left\{
        \begin{matrix}
            \frac{1}{\theta}, & x = 1,2,\ldots \theta \\
            0,                & \text{otherwise}
        \end{matrix}
        \right. .
    \end{equation*}
    The restriction $x_i \in \left\{ 1, \ldots ,\theta \right\}$ for $i=1,\ldots ,n$ can be re-expressed as $x_i \in \left\{ 1,2 , \ldots \right\}$ for $i = 1,\ldots n$ (note that there is no $\theta$ in this restriction) and $\max_i x_i \leq \theta$. If we define $T(\bm{x}) = \max_i x_i = x_{(n)}$,
    \begin{equation*}
        h(x) =
        \left\{
        \begin{matrix}
            1, & x_i \in \left\{ 1, \ldots ,\theta \right\} \; \text{for} \; i=1,\ldots ,n \\
            0, & \text{otherwise}
        \end{matrix}
        \right.
    \end{equation*}
    and
    \begin{equation*}
        g(t \mid \theta) =
        \left\{
        \begin{matrix}
            \theta^{-n} & t \leq \theta    \\
            0,          & \text{otherwise}
        \end{matrix}
        \right. .
    \end{equation*}
    It is easily verified that $f(\bm{x} \mid \theta) = g(T(\bm{x}) \mid \theta)$ for all $\bm{x}$ and $\theta$. Thus, according to \Cref{thm: factorization_theorem}, the largest order statistic, $T(\bm{X}) = X_{(n)}$, is a sufficient statistic in this problem. This type of analysis can sometimes be carried out more clearly and concisely using indicator function. Let $\NN$ be the set of natural numbers (discluding $0$) and $\NN_{\theta}$ be the natural numbers up to and including $\theta$. Then the joint pmf of $X_1 , \ldots , X_n$ is
    \begin{equation*}
        f(\bm{x} \mid \theta) = \prod_{i=1}^{n} \theta^{-1} \Id_{N_\theta} (x_i) = \theta^{-n} \prod_{i=1}^{n} \Id_{N_\theta} (x_i).
    \end{equation*}
    Defining $T(\bm{x}) = x_{(n)}$, we see that
    \begin{equation*}
        \prod_{i=1}^{n} \Id_{N_\theta} (x_i) = \left( \prod_{i=1}^{n} \Id_{N} (x_i) \right) \Id_{N_\theta} (T(x))
    \end{equation*}
    thus providing the factorization
    \begin{equation*}
        f(\bm{x} \mid \theta) = \theta^{-n} \Id_{N_\theta} (T(x)) \left( \prod_{i=1}^{n} \Id_{N} (x_i) \right).
    \end{equation*}
    The first factor depends on $x_1 , \ldots , x_n$ only through the value of $T(\bm{x}) = x_{(n)}$, and the second factor does not depend on $\theta$. Again, according to \Cref{thm: factorization_theorem}, $T(\bm{X}) = X_{(n)}$, is a sufficient statistic in this problem.
\end{exam}

\begin{defe}[Likelihood, Log-Likelihood and Score Function] \label{defe: likelihood_function}
    Let $f(\bm{x} \mid \theta)$ denote the joint pdf or pmf of the sample $\bm{X} = (X_1 , \ldots , X_2)$. Then, given that $\bm{X} = \bm{x}$ is observed, the function of $\theta$ defined by
    \begin{equation*}
        L(\theta \mid \bm{x}) = f(\bm{x} \mid \theta)
    \end{equation*}
    is called the {\bf likelihood function} \cite{CasellaGeorge2001SI}*{page 290}. For a given outcome $\bm{x}$ of $\bm{X}$, the {\bf log-likelihood function}, denoted $l$, is the natural logarithm of the likelihood function
    \begin{equation*}
        l(\theta \mid \bm{x}) = \ln L(\theta \mid \bm{x}) = \ln f(\bm{x} \mid \theta).
    \end{equation*}
    It's gradient with respect to $\theta$, denoted $S$, is called the {\bf score function}
    \begin{equation*}
        S (\theta \mid \bm{x}) = \nabla_{\theta} \, l (\theta \mid \bm{x}) \frac{\nabla_{\theta} f(\bm{x} \mid \theta)}{f(\bm{x} \mid \theta)}
    \end{equation*}
    \cite{KroeseDirkP2013SMaC}*{page 165}.
\end{defe}

\begin{defe}[Expotential Family] \label{defe: exp_fam}
    In the case of $p-$dimensional observation $\bm{x}_1 , \bm{x}_2 , \ldots , \bm{x}_n \in \CC^{p}$, a $d-$dimensional parameter vector $\bm{\theta} \in \CC^{d}$, and a $q-$dimensional sufficient statistic $T(\bm{x}_1 , \ldots , \bm{x}_n) \in \CC^{q}$, the likelihood function $L (\bm{\theta})$ for the $d-$parameter vector $\bm{\theta}$ has the following form if it belongs to the $d-$parameter {\bf exponential family}
    \begin{equation*}
        L (\bm{\theta}) = b(\bm{x}_1 , \ldots , \bm{x}_n) \exp \left\{ c(\bm{\theta})^{\intercal} T(\bm{x}_1 , \ldots , \bm{x}_n) \right\} / a(\bm{\theta})
    \end{equation*}
    where $c(\bm{\theta}) \in \CC^{q}$ and $b(\bm{x}_1 , \ldots , \bm{x}_n)$ and $a(\bm{\theta})$ are scalar functions \cite{CasellaGeorge2001SI}*{page 279}.
\end{defe}

\begin{thm} \label{thm: exp_fam_ss}
    Let $\bm{X}_1 , \bm{X}_2 , \ldots , \bm{X}_n$ be iid observations from a pdf or pmf $f(x \mid \bm{\theta})$ that belongs to an exponential family as seen in \Cref{defe: exp_fam}, then
    \begin{equation*}
        T(\bm{X}_1 , \ldots , \bm{X}_n) = \left( \sum_{j=1}^{n} t_1 (\bm{X}_j) , \ldots , \sum_{j=1}^{n} t_k (\bm{X}_j) \right)
    \end{equation*}
    is a sufficient statistic for $\bm{\theta}$ \cite{CasellaGeorge2001SI}*{page 279}.
\end{thm}

\begin{defe}[Minimal Sufficient Statistic] \label{defe: minimal_sufficient_statistic}
    A sufficient statistic $T(\bm{X})$ is called a {\bf minimal sufficient statistic} if, for any other sufficient statistic $T'(\bm{X})$, $T(\bm{x})$ is a function of $T'(\bm{x})$ \cite{CasellaGeorge2001SI}*{page 280}.
\end{defe}

\begin{thm} \label{thm: mss_ratio_test}
    Let $f(\bm{x} \mid \theta)$ be the pd of a sample $\bm{X}$. Suppose there exists a function $T(\bm{x})$ such that, for every two sample points $\bm{x}$ and $\bm{y}$, the ratio $f(\bm{x} \mid \theta) / f(\bm{y} \mid \theta)$ is constant as a function of $\theta$ if and only if $T(\bm{x}) = T(\bm{y})$. Then $T(\bm{X})$ is a minimal sufficient statistic \cite{CasellaGeorge2001SI}*{page 281}.
\end{thm}

\begin{exam}[Normal Minimal Sufficient Statistic] \label{exam: norm_min_ss}
    Example taken from \cite{CasellaGeorge2001SI}*{page 281}. Let $X_1 , \ldots , X_n \iid \Nor (\mu , \sigma^2)$, where both $\mu$ and $\sigma^2$ unknown. Let $\bm{x}$ and $\bm{y}$ denote two sample points, and let $(\overline{x}, s_{\bm{x}}^2)$ and $(\overline{y}, s_{\bm{y}}^2)$ be the sample means and variances corresponding to the $\bm{x}$ and $\bm{y}$ samples, respectively. Then, the ratio of the densities becomes
    \begin{align*}
        \frac{f\left(\mathbf{x} \mid \mu, \sigma^{2}\right)}{f\left(\mathbf{y} \mid \mu, \sigma^{2}\right)} \
         & =\frac{\left(2 \pi \sigma^{2}\right)^{-n / 2} \exp \left(-\left[n(\bar{x}-\mu)^{2}+(n-1) s_{\mathbf{x}}^{2}\right] /\left(2 \sigma^{2}\right)\right)}{\left(2 \pi \sigma^{2}\right)^{-n / 2} \exp \left(-\left[n(\bar{y}-\mu)^{2}+(n-1) s_{\mathbf{y}}^{2}\right] /\left(2 \sigma^{2}\right)\right)} \\
         & =\exp \left(\left[-n\left(\bar{x}^{2}-\bar{y}^{2}\right)+2 n \mu(\bar{x}-\bar{y})-(n-1)\left(s_{\mathbf{x}}^{2}-s_{\mathbf{y}}^{2}\right)\right] /\left(2 \sigma^{2}\right)\right) .
    \end{align*}
    This ratio will be constant as a function of $\mu$ and $\sigma^2$ if and only if $\overline{x}=\overline{y}$ and $s_{\bm{x}}^2 = s_{\bm{y}}^2$. Thus by \Cref{thm: mss_ratio_test}, $(\overline{X}, S^2)$ is a minimal sufficient statistic for $(\mu , \sigma^2 )$.
\end{exam}

\begin{defe}[Ancillary Statistic] \label{defe: ancillary_statistic}
    A statistic $S(\bm{X})$ whose distribution does not depend on the parameter $\theta$ is called an ancillary statistic \cite{CasellaGeorge2001SI}*{page 282}.
\end{defe}

\begin{defe}[Complete Distributions and Statistics] \label{defe: complete_statistic}
    Let $f(t \mid \theta)$ be a family of pdfs or pmfs for a statistic $T(\bm{X})$. The family of probability distributions is called {\bf complete} if $\EE_{\theta} g(T) = 0$ for all $\theta$ implies $\PP (g(T) = 0) = 1$ for all $\theta$. Equivalently, $T(\bm{X})$ is called a {\bf complete statistic} \cite{CasellaGeorge2001SI}*{page 285}.
\end{defe}

\begin{exam}[Binomial Complete Statistic] \label{exam: bin_comp_stat}
    Example taken from \cite{CasellaGeorge2001SI}*{page 285}. Suppose that $T$ has a $\Bin (n,p)$ distribution, $0<p<1$. Let $g$ be a function such that $\EE_{p} g(T) = 0$. Then
    \begin{align*}
        0 = \EE_{p} g(T) & = \sum_{t=0}^{n} g(t) \binom{n}{t} p^{t} (1-p)^{n-1}                        \\
                         & = (1-p)^n \sum_{t=0}^{n} g(t) \binom{n}{t} \left( \frac{p}{1-p} \right)^{t}
    \end{align*}
    for all $p$, $0<p<1$. The factor $(1-p)^n$ is not 0 for any $p$ in this range. Thus it must be that
    \begin{equation*}
        0 = \sum_{t=0}^{n} g(t) \binom{n}{t} \left( \frac{p}{1-p} \right)^{t} = \sum_{t=0}^{n} g(t) \binom{n}{t} r^t
    \end{equation*}
    for all, $0 < r < \infty$. But the last expression is a polynomial of degree $n$ in $r$, where the coefficient of $r^t$ is $g(t) \binom{n}{t}$. For the polynomial to be $0$ for all $r$, each coefficient must be $0$. Since none of the $\binom{n}{t}$ terms is $0$, this implies that $g(t) = 0$ for $t=0,1,\ldots n$. Since $T$ takes on the values $0,1,\ldots n$ with probability $1$, this means that $\PP_p (g(T) = 0) = 1$ for all $p$, the desired conclusion. Hence, $T$ is a complete statistic.
\end{exam}

\begin{defe}[Point Estimator] \label{defe: point_estimator}
    A {\bf point estimator} is any function $W(X_1 , \ldots , X_n)$ of a sample; that is, any statistic (see \Cref{defe: statistic}) is a point estimator \cite{CasellaGeorge2001SI}*{page 311}.
\end{defe}